\documentclass[11pt]{article}
\usepackage[margin=1in]{geometry}
\usepackage{amsmath}
\usepackage{graphicx}
\usepackage{enumitem}
\usepackage{fancyhdr}
\usepackage{tikz}
\usepackage{array}
\usepackage{multirow}
\usepackage{colortbl}
\usepackage[utf8]{inputenc}
\usepackage[spanish,es-noquoting]{babel}

% Header setup
\pagestyle{fancy}
\fancyhf{}
\lhead{Nombre: \underline{\hspace{3cm}} \quad Período: \underline{\hspace{1cm}}}
\rhead{Fecha: \underline{\hspace{2.5cm}}}
\cfoot{\thepage}
\renewcommand{\headrulewidth}{0.5pt}

% Custom commands
\newcommand{\blank}[1]{\underline{\hspace{#1}}}

\title{\vspace{-2cm}\textbf{De la Subasta de Dulces al Ingreso Máximo}\\
\large Encontrando el Punto Óptimo: Optimización de Precios a Través de Datos}
\author{}
\date{}

\begin{document}
\maketitle
\thispagestyle{fancy}

\section*{Parte 1: Revisando Nuestros Datos de la Subasta}

Ayer, realizamos una subasta de dulces donde cada grupo podía comprar tantos dulces como quisiera a cada precio. Organicemos lo que descubrimos sobre la \textbf{demanda} -- cuántos dulces querían comprar las personas a diferentes precios.

\subsection*{Paso 1: Registra los Resultados de Tu Subasta}

Completa la tabla con los datos de nuestras 5 rondas de subasta:

\begin{center}
\renewcommand{\arraystretch}{2}
\begin{tabular}{|c|c|c|}
\hline
\rowcolor{gray!20}
\textbf{Ronda} & \textbf{Precio (\$)} & \textbf{Cantidad Total Demandada} \\
& & \textit{(dulces comprados por todos los grupos)} \\
\hline
1 & \blank{2cm} & \blank{3cm} \\
\hline
2 & \blank{2cm} & \blank{3cm} \\
\hline
3 & \blank{2cm} & \blank{3cm} \\
\hline
4 & \blank{2cm} & \blank{3cm} \\
\hline
5 & \blank{2cm} & \blank{3cm} \\
\hline
\end{tabular}
\end{center}

\subsection*{Paso 2: Verificación Rápida de la Demanda}

\textbf{Pregunta:} Observa tus datos de Precio vs. Cantidad Demandada. ¿Qué patrón notas?

\vspace{1cm}
\blank{12cm}

\vspace{0.5cm}
\blank{12cm}

¡Este patrón se llama la \textbf{Ley de la Demanda} en economía!

\section*{Parte 2: De la Demanda al Ingreso}

Ahora viene la perspicacia empresarial: No se trata solo de cuánto vendemos -- ¡se trata de cuánto dinero ganamos!

\subsection*{Paso 3: Calcula el Ingreso}

Para cada ronda, calcula el \textbf{ingreso} (dinero total recaudado):

\begin{center}
\colorbox{yellow!30}{
\textbf{Ingreso = Precio $\times$ Cantidad Demandada}
}
\end{center}

\vspace{0.5cm}

\begin{center}
\renewcommand{\arraystretch}{2}
\begin{tabular}{|c|c|c|c|}
\hline
\rowcolor{gray!20}
\textbf{Ronda} & \textbf{Precio (\$)} & \textbf{Cantidad Demandada} & \textbf{Ingreso (\$)} \\
& \textit{(copiar de arriba)} & \textit{(copiar de arriba)} & \textit{(calcular: Precio $\times$ Cantidad)} \\
\hline
1 & \blank{2cm} & \blank{2cm} & \blank{3cm} \\
\hline
2 & \blank{2cm} & \blank{2cm} & \blank{3cm} \\
\hline
3 & \blank{2cm} & \blank{2cm} & \blank{3cm} \\
\hline
4 & \blank{2cm} & \blank{2cm} & \blank{3cm} \\
\hline
5 & \blank{2cm} & \blank{2cm} & \blank{3cm} \\
\hline
\end{tabular}
\end{center}

\subsection*{Paso 4: La Pregunta del Millón}

Mirando tu columna de ingresos, ¿qué precio te dio el \textbf{ingreso más alto}?

Precio: \$\blank{2cm} \qquad Ingreso a ese precio: \$\blank{3cm}

\section*{Parte 3: Creando la Gráfica de Ingresos en Desmos}

\subsection*{Paso 5: Grafica Precio vs. Ingreso}

\begin{enumerate}[label=\alph*)]
\item Abre la calculadora gráfica Desmos
\item Crea una tabla nueva con dos columnas:
    \begin{itemize}
    \item Columna 1: Etiquétala $x_1$ (este es el Precio)
    \item Columna 2: Etiquétala $y_1$ (este es el Ingreso)
    \end{itemize}
\item Ingresa tus 5 puntos de datos de las columnas de Precio e Ingreso de arriba
\item ¡Deberías ver puntos formando una curva!
\end{enumerate}

\textbf{Dibuja lo que ves:}

\begin{center}
\begin{tikzpicture}[scale=0.8]
% Grid
\draw[gray!30, very thin] (0,0) grid (10,6);
% Axes
\draw[thick, ->] (-0.5,0) -- (10.5,0) node[right] {Precio (\$)};
\draw[thick, ->] (0,-0.5) -- (0,6.5) node[above] {Ingreso (\$)};
% Add axis labels
\foreach \x in {0,2,4,6,8,10}
    \node[below] at (\x,0) {\small \x};
\foreach \y in {0,1,2,3,4,5,6}
    \node[left] at (0,\y) {\small \y};
\end{tikzpicture}
\end{center}

\section*{Parte 4: Encontrando el Precio Perfecto con Regresión}

\subsection*{Paso 6: Ajusta una Cuadrática (Parábola) a Tus Datos}

En Desmos, escribe esto en una línea nueva:
\begin{center}
\colorbox{blue!10}{\texttt{$y_1 \sim ax_1^2 + bx_1 + c$}}
\end{center}

¡Desmos encontrará automáticamente la parábola que mejor se ajuste! Escribe la ecuación que te da:

\vspace{0.5cm}
\begin{center}
\fbox{
$I(p) = \blank{2cm} p^2 + \blank{2cm} p + \blank{2cm}$
}
\end{center}

Registra los valores que Desmos calculó:
\begin{itemize}
\item $a = $ \blank{3cm} (esto debe ser negativo porque nuestra parábola abre hacia abajo y tiene un máximo)
\item $b = $ \blank{3cm}
\item $c = $ \blank{3cm}
\end{itemize}

\section*{Parte 5: El Vértice -- ¡Tu Precio Óptimo!}

El punto más alto en tu parábola de ingresos se llama el \textbf{vértice}. ¡Esto te dice el mejor precio para cobrar!

\subsection*{Paso 7: Encuentra el Vértice Gráficamente}

\begin{enumerate}[label=\alph*)]
\item Haz clic en el punto más alto de tu parábola en Desmos
\item Desmos te mostrará las coordenadas: (Precio, Ingreso)
\end{enumerate}

\textbf{Vértice de la Gráfica:} \quad Precio = \$\blank{2cm} \quad Ingreso Máximo = \$\blank{3cm}

\subsection*{Paso 8: Encuentra el Vértice Usando Álgebra}

Para cualquier parábola $y = ax^2 + bx + c$, la coordenada x del vértice es:

\begin{center}
\colorbox{yellow!30}{
$x = -\dfrac{b}{2a}$
}
\end{center}

Usando tus valores de $a$ y $b$ del Paso 6:

\vspace{0.5cm}
\begin{align*}
\text{Precio Óptimo} &= -\dfrac{b}{2a}\\[0.5cm]
&= -\dfrac{\blank{3cm}}{2 \times (\blank{3cm})}\\[0.5cm]
&= -\dfrac{\blank{3cm}}{\blank{3cm}}\\[0.5cm]
&= \$\blank{3cm}
\end{align*}

\textbf{Verifica:} ¿Coincide esto con lo que encontraste gráficamente? \quad Sí \quad No

\section*{Parte 6: Dando Sentido Empresarial}

\subsection*{Paso 9: Interpreta Tus Resultados}

Completa estas oraciones basándote en tu análisis:

\begin{enumerate}[label=\alph*)]
\item Si ponemos el precio del dulce muy bajo (como \$\blank{1.5cm}), vendemos mucho pero no ganamos mucho por dulce,
así que nuestro ingreso total es \blank{5cm}.

\item Si ponemos el precio del dulce muy alto (como \$\blank{1.5cm}), ganamos mucho por dulce pero \blank{5cm},
así que nuestro ingreso también es bajo.

\item El precio ``punto óptimo'' que maximiza nuestro ingreso es \$\blank{1.5cm},
lo que nos daría un ingreso de \$\blank{2cm}.
\end{enumerate}

\subsection*{Paso 10: Recomendación Empresarial}

Escribe una recomendación de 2-3 oraciones para el dueño de una tienda de dulces basándote en tu análisis:

\vspace{1cm}
\hrule
\vspace{1cm}
\hrule
\vspace{1cm}
\hrule
\vspace{1cm}
\hrule

\section*{Pregunta de Extensión (si hay tiempo)}

\textbf{Piénsalo:} Encontramos el precio que maximiza el \textit{ingreso}, pero a las empresas les importa la \textit{ganancia}.
¿Qué información adicional necesitaríamos para encontrar el precio que maximiza la ganancia en lugar del ingreso?

\vspace{1cm}
\blank{12cm}

\vspace{0.5cm}
\blank{12cm}

\vspace{2cm}

\begin{center}
\rule{0.8\textwidth}{0.5pt}

\textit{Próxima Clase: ¡Aplicaremos estas mismas técnicas para ayudar a negocios locales reales (Golden Monkey, Mandee's y Yas Chicken) a encontrar sus precios óptimos!}
\end{center}

\end{document}